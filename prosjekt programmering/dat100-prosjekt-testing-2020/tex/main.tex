\documentclass[]{article}

\usepackage{mathptm}
\usepackage{xspace}
\usepackage{amsmath}
\usepackage{graphicx}
\graphicspath{ {figs/} }
\usepackage{algorithm}
\usepackage{algpseudocode}
\usepackage{tikz}
\usepackage{tkz-graph}
%\bibliographystyle{splncs03}
\usetikzlibrary{shapes.misc, positioning}
\DeclareMathOperator{\atantwo}{atan2}
\begin{document}

\section*{Haversine Formlen}

\begin{align*}
  \mbox{GPS posisjon 1} & : (\mbox{lattitude}_1,\mbox{longitude}_1) \\
  \mbox{GPS posisjon 2} & : (\mbox{lattitude}_2,\mbox{longitude}_2) \\
\end{align*}

\begin{align*}
R & = 6371000 \mbox{ meter (jordens gjennomsnitsradius)} \\
\end{align*}



\begin{align*}
  \varphi_1 & = \mbox{lattitude}_1 \mbox{ (omregnet til radianer)} \\
  \varphi_2 & = \mbox{latitude}_2 \mbox{ (omregnet til radianer)} \\
  \Delta \varphi & = \mbox{latitude}_2 - \mbox{latitude}_1 \mbox{ (omregnet til radianer)} \\
  \Delta \lambda &  = \mbox{longitude}_2 - \mbox{longitude}_1 \mbox{ (omregnet til radianer)} \\
  a & = (\sin{(\Delta \varphi / 2)})^2 + \cos{\varphi_1} \cdot \cos{\varphi_2} \cdot (\sin{(\Delta \lambda / 2))^2} \\
  c & = 2 \cdot \atantwo(\sqrt{a},\sqrt{(1-a)} \\
  d & = R \cdot c 
\end{align*}

\end{document}
